\section{Circuito integratore}
\begin{figure}[h]
	\centering
	\includegraphics[scale=1]{integratore.pdf}
	\caption{circuito integratore con OpAmp.}
	\label{f:integratore}
\end{figure}
Si è montato il circuito in \ref{f:integratore} con $R_1= \SI{0.981(9)}{\kohm}$, $R_2= \SI{9.87(9)}{\kohm}$,$C_1= 48\pm2$nF. L'ampiezza picco-picco di $V_{in}=\SI{2.08(2)}{\V}$. Al variare della frequenza si è misurato $V_{out}$ con l'oscilloscopio. La frequenza è stata misurata con il frequenzimetro dell'oscilloscopio e lo sfasamento tra $V_{in}$ e $V_{out}$ si è ricavato dalla misura dell'intervallo di tempo $\Delta$T tra le due intersezioni delle onde in ingresso e uscita con l'asse delle ascisse \footnote{Tale asse orizzontale corrispomde per ogni onda ad una tensione costante pari al proprio valor medio}.Da questa misura si ricava lo sfasemento:$ \Delta\phi = 2\Delta$Tf.\\
Per quanto riguarda il guadagno in frequenza sono stati eseguiti due fit(in \ref{f:guad_integ}), uno nella parte piatta dei dati cioè a basse frequenze ed un altro ad alte frequenze per studiare i due limiti del circuito integratore, rispettivamente $f<<f_t$ e $f>>f_t$.
Per $f_t$ si intende la frequenza di taglio del circuito integratore pari a $f_t=\frac{1}{2\pi R_2C_1}= \SI{335(16)}{\Hz}$. \\
\\
Il fit a basse frequenze ($f< \SI{50}{\Hz}$) è stato eseguito con una costante e i risultati sono :\\
$A_v= \SI{20.05(2)}{}$\\
$\chi^2=4.79$ ($4$ dof, $p = 0.31$)\\
\\
Il fit ad alte frequenze ($f> \SI{2}{\kHz}$) è stato eseguito con una funzione lineare $A_v(dB)= a\log_{10} f +b$  e i risultati sono:\\
$a= \SI{-19.8(2)}{\frac{\dB}{\deca}}$\\
$b= \SI{69.9(4)}{\dB}$\\
$\chi^2=3.89$ ($5$ dof, $p = 0.56$)\\
\\
Il valore atteso del guadagno a basse frequenze è $A_v=20\log_{10} \frac{R_2}{R_1}= \ SI{20.1(2)}{dB}$ compatibile con il valore ottenuto dal fit.
Ad alte frequenze la pendenza della retta è compatibile con $\SI{-20}{\frac{\dB}{\deca}}$.\\
E' stato eseguito anche un fit allo sfasamento() con un modello non lineare $\Delta \phi= \arctan{\frac{-f}{f_t}}$ e si è ottenuto:\\
$f_t= \SI{321(2)}{\Hz}$\\
$\chi^2=62.21$ ($16$ dof, $p = 0$)\\
Il valore della frequenza di taglio risulta compatibile con quello atteso prima calcolato.\\
\\
Si è poi verificata la risposta del circuito ad un'onda quadra di frequenza $f= \SI{10.6(1)}{\kHz}$. Con un'ampiezza di $V_{in}=\SI{3.63(2)}{\V}$ si è ottenuta un'ampiezza di $V_{out}=\SI{1.90(2)}{\V}$ quindi $A_v=\SI{-5.6(2)}{\dB}$.





\begin{figure}[h]
	\centering
	\includegraphics[scale=0.7]{fit_guad_integratore.pdf}
	\caption{plot di bode del guadagno del circuito integratore}
	\label{f:guad_integ}
\end{figure}

\begin{figure}[h]
	\centering
	\includegraphics[scale=0.7]{fit_fase_integratore.pdf}
	\caption{fase in unità $\pi$ del circuito integratore in funzione della frequenza}
	\label{f:fase_integ}
\end{figure}