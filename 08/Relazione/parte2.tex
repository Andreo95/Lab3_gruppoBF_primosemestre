\section{Caratteristiche dell'oscillatore}
Si è ricollegato il punto A all'ingresso non-invertente dell'OpAmp disconnettendo il generatore.Si nota che se la resistenza del potenziometro verso $R_5$ aumenta molto allora non si avrà il segnale oscillante atteso all'uscita $V_A$ . Infatti l'oscillazione si innesca quando la resistenza (prima considerata del potenziometro) scende sotto il 30 \% di quella totale. 

Si misura ora la frequenza di oscillazione $f= \SI{1.600(5)}{\kHz}$. Variando la tensione di alimentazione dell'OpAmp non si osservano variazioni della frequenza. Variando la resistenza del potenziometro non ci sono variazioni della frequenza ,sempre mantenendosi nei limiti di resistenza prima descritti.

Si è regolata la resistenza del potenziometro in modo da trovare il punto di innesco dell'oscillazione. Mantenendo tale settaggio, si è scollegato nuovamente il punto A dall'ingresso non-invertente ed inviando un segnale all'ingresso di ampiezza $V_+=\SI{262(2)}{\mV}$ si è ottenuto $V_{out}= \SI{764(4)}{\mV}$ , da cui si ricava $A=\SI{2.92(4)}{}$ che è compatibile con il valore atteso pari a $A_{atteso}= 1+\frac{R_1}{R_2}(1+\frac{\omega_s}{\omega_p})= \SI{2.9(1)}{}$.